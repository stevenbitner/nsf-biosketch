\documentclass[svgnames,11pt]{article}
\usepackage[margin=1in]{geometry}
\usepackage[T1]{fontenc}
\usepackage{times}
\usepackage{float}
\usepackage{hyperref}
\usepackage{etoolbox}
\usepackage[english]{babel}
\patchcmd{\thebibliography}{\section*{\refname}}{}{}{}

\title{%
        \vspace{-2\baselineskip}
            \normalsize
            {\textbf{Steven P. Bitner}}\\
            \vspace{0.5\baselineskip}
            \hrule
            \vspace{0.5\baselineskip}
            Department of Computer Science, The University of West Florida, Pensacola, FL-32514 \\
            \textit{e-mail:} sbitner@uwf.edu,
            \textit{phone}: +1 (850) 473-7063
        \vspace{-1.5ex}
}        
\date{}
\begin{document}
\maketitle
\vspace{-4\baselineskip}
\subsection*{(a) Professional Preparation}
\begin{table}[H]
\centering
\begin{tabular}{llr}
\hline
Institute                                                & Major          & Degree, Year            \\ \hline
Texas State University - San Marcos & Computer Science    & BS, 2006       \\
The University of Texas at Dallas & Computer Science  & MS, 2008       \\
The University of Texas at Dallas & Computer Science  & PhD, 2010       \\
\end{tabular}
\end{table}

\subsection*{(b) Appointments}
\begin{table}[H]
\centering
\begin{tabular}{ll}
2018-present & Assistant Professor, Department of Computer Science, University of West Florida \\
2015-present   & Chief Software Architect, First Principle Innovators, San Francisco, CA \\
2013-2015   & Senior Software Engineer, Answers Corporation, Saint Louis, MO \\
2010-2013   & Simulation Software Developer, US Army TRADOC Analysis Center, Fort Leavenworth, KS \\
2006-2010   & Research Assistant, University of Texas at Dallas
\end{tabular}
\end{table}
\subsection*{(c) Products}

\nocite{DBLP:journals/comgeo/BitnerD12}
\bibliographystyle{acm}
\bibliography{biosketch}








TODO 








\subsection*{(d) Synergistic Activities}
\begin{enumerate}
\item Publicly available, pipeline and scripts development (https://github.com/ISUgenomics and \\
https://github.com/aseetharam). Some examples include:
\begin{itemize}
\item common\_analyses: repository for most commonly used programs optimized to run on clusters
\item common\_scripts: repository for simple utility scripts to efficiently run or manipulate the NGS data.
\item StampedeBLAST: Optimized NCBI-BLAST program pipeline for the XSEDE (Stampede) cluster.
\item basic\_UNIX\_2015: workshop materials for the basic UNIX workshop, offered during summer, 2015 at Iowa State University.
\end{itemize}
\item Publicly available tutorials for NGS analyses (http://gif.biotech.iastate.edu/Tutorial): Actively contributed for developing a comprehensive wiki pages for performing NGS data analyses. The pages include many custom pipelines as well as custom scripts that can be readily used for specific problems.

\item Guest lecturer: in the courses related to Phylogenomics/evolutionary biology (EEOB 561X, spring 2014, ENTM595 during fall 2013) and Next Generation Sequencing (GEN 349X during spring 2014). 

\item Workshop instructor: Co-organized and instructed summer workshops on UNIX skills at Iowa State University. The audience included staff, faculty and various graduate students. This 4 hour crash course is conducted every year with 2 sessions during the summer months. I also instructed a one-day hands-on UNIX workshop (multiple times) for faculty and students of Purdue University (over 100 attendee's total). Developed teaching materials, data and slides for the interactive class session.

\item Co-instructor: Co-taught graduate level course in Bioinformatics (BIO487/587, fall 2012) program. Shared responsibility for preparing lectures, exams, homework assignments, and grading. Taught under-graduate level lab in Genetics (BIO382L, fall 2012) program. Full responsibility for lectures, exams, homework assignments, and grading.
\end{enumerate}
\end{document}
